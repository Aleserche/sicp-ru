\chapter*{Предисловие ко второму изданию}
\markboth{Предисловие ко второму изданию}{Предисловие ко второму изданию}
\addcontentsline{toc}{chapter}{Предисловие ко второму изданию}
\thispagestyle{empty}

\epigraph{
Возможно ли, что программы не похожи ни на что
другое, что они предназначены на выброс; что вся
штука состоит в том, чтобы всегда видеть в них мыльный
пузырь?}{Алан Дж. Перлис}
\index{ru}{Перлис, Алан~Дж.||Alan~J. Perlis||n|}\index{en}{Alan~J. Perlis||Перлис, Алан~Дж.||n|}
Материал этой книги был основой вводного курса по
информатике в MIT начиная с 1980 года.  К тому времени, как было
выпущено первое издание, мы преподавали этот материал в течение четырех
лет, и прошло еще двенадцать лет до появления второго
издания.  Нам приятно, что наша работа была широко признана и включена в
другие тексты.  Мы видели, как наши ученики черпали идеи и программы из
этой книги и на их основе строили новые компьютерные системы и
языки.  Буквально по старому талмудическому каламбуру, наши ученики
стали нашими строителями.  Мы рады, что у нас такие одаренные ученики и
такие превосходные строители.

Готовя это издание, мы включили в него сотни
поправок, которые нам подсказали как наш собственный преподавательский 
опыт, так и советы коллег из MIT и других мест.  Мы заново
спроектировали большинство основных программных систем в этой книге,
включая систему обобщенной арифметики, интерпретаторы, имитатор
регистровых машин и компилятор; кроме того, мы переписали все примеры
программ так, чтобы любая реализация Scheme, соответствующая стандарту
Scheme IEEE (IEEE 1990), была способна выполнять этот код.

В этом издании подчеркиваются несколько новых тем.  Самая
важная из них состоит в том, что центральную роль в вычислительных
моделях играют различные подходы
ко времени: объекты, обладающие состоянием,
параллельное программирование, функциональное программирование,
ленивые вычисления и недетерминистское программирование.  Мы включили
в текст новые разделы по параллельным вычислениям и недетерминизму и
постарались интегрировать эту тему в материал книги на всем ее
протяжении.

Первое издание книги почти точно следовало программе нашего
односеместрового курса в MIT.  Рассмотреть весь материал, включая то,
что добавлено во втором издании, в течение семестра будет невозможно, так что
преподавателю придется выбирать.  В нашей собственной практике мы
иногда пропускаем раздел про логическое программирование 
(раздел~\ref{LOGIC-PROGRAMMING}); наши студенты
используют имитатор регистровых машин, но мы не описываем его
реализацию (раздел~\ref{A-REGISTER-MACHINE-SIMULATOR}); наконец, мы 
даем лишь беглый обзор компилятора (раздел~\ref{COMPILATION}). 
Даже в таком виде курс остается интенсивным. Некоторые преподаватели 
предпочтут ограничиться первыми тремя или четырьмя главами, оставляя
прочий материал для последующих курсов.

Сайт World Wide Web \url{http://mitpress.mit.edu/sicp}
предоставляет поддержку
пользователям этой книги.  Там есть программы из книги, простые
задания по программированию, сопроводительные материалы и реализации
диалекта Лиспа Scheme.\translationnote{В настоящее время (август 2005~г.) 
на сайте имеется также полный текст англоязычного издания.
}
{\sloppy

}
