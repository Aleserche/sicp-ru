\chapter*{Благодарности}
\markboth{Благодарности}{Благодарности}
\addcontentsline{toc}{chapter}{Благодарности}
\thispagestyle{empty}
Мы хотели бы поблагодарить множество людей, которые помогли
нам создать эту книгу и этот курс.

Наш курс --- очевидный интеллектуальный потомок <<6.321>>,
замечательного курса по компьютерной лингвистике и лямбда-исчислению,
который читали в MIT в конце 60-х Джек Уозенкрафт и Артур Эванс~мл.

Мы очень обязаны Роберту Фано, который реорганизовал вводную 
программу MIT по электротехнике и информатике, сосредоточившись на
принципах технического проектирования.  Он вдохновил нас на это
предприятие и написал первую программу курса, из которого развилась
эта книга.

Стиль и эстетика программирования, которые мы
пытаемся привить читателю, во многом были разработаны совместно с Гаем
Льюисом Стилом~мл., 
который вместе с Джеральдом Джеем Сассманом участвовал в
первоначальной разработке языка Scheme. В дополнение к этому Дэвид
Тёрнер, Питер Хендерсон, Дэн Фридман, Дэвид Уайз и Уилл Клингер
научили нас многим из приемов функционального программирования,
которые излагаются в данной книге.

Джон Мозес научил нас структурировать большие системы.
Благодаря его опыту с системой символьных вычислений Macsyma мы стали
понимать,  
что необходимо избегать усложненности структур управления и в первую очередь
заботиться о такой организации данных, которая  отражает
реальную структуру моделируемого мира

Марвин Минский и Сеймур Пэйперт сильно повлияли на формирование
нашего подхода к программированию и к его месту в нашей интеллектуальной
жизни.  Благодаря им мы понимаем, что вычисление дает нам 
средство выражения и исследования мыслей, которые иначе были бы
слишком сложны, чтобы с ними можно было точно работать.  Они
подчеркивают, что способность писать и изменять
программы дает студенту мощное средство, с помощью которого исследование
становится естественной деятельностью.

Кроме того, мы полностью согласны с Аланом Перлисом в том, что
программирование~--- это огромное удовольствие и что нам нужно стараться
поддерживать радость программирования.  Часть этой
радости приходит от наблюдения за работой великих мастеров.  Нам
выпало счастье быть учениками у ног Билла Госпера и Ричарда
Гринблатта.

Трудно перечислить всех тех, кто принял участие в
развитии программы нашего курса.  Мы благодарим всех лекторов,
инструкторов и тьюторов, которые работали с нами в прошедшие
пятнадцать лет и потратили много часов сверхурочной работы на наш предмет,
особенно Билла Сиберта, Альберта Мейера, Джо Стоя, Рэнди Дэвиса, Луи
Брэйда, Эрика Гримсона, Рода Брукса, Линна Стейна и Питера
Соловитца. Мы бы хотели особо отметить выдающийся педагогический вклад 
Франклина Турбака, который теперь преподает в Уэллесли: его работа по
обучению младшекурсников установила стандарт, на который мы все
можем равняться.  Мы благодарны Джерри Сальтцеру и Джиму Миллеру,
которые помогли нам бороться с тайнами параллельных вычислений, а
также Питеру Соловитцу и Дэвиду Макаллестеру за их вклад в
представление недетерминистских вычислений в 
главе~\ref{METALINGUISTIC-ABSTRACTION}.

Много людей вложило немалый труд в преподавание этого
материала и в других университетах.  Вот некоторые из тех, с кем мы тесно
общались в работе: это Джекоб Кацнельсон в Технионе, Хэрди Майер в
Калифорнийском университете в Ирвине, Джо Стой в Оксфорде, Элиша Сэкс
в университете Пердью и Ян Коморовский в Норвежском университете Науки 
и Техники.  Мы гордимся коллегами, которые получили
награды за адаптацию этого предмета в других университетах: это
Кеннет Йип в Йеле, Брайан Харви в Калифорнийском университете в
Беркли и Дон Хаттенлохер в Корнелле.

Эл Мойе дал нам возможность прочитать этот материал инженерам 
компании Хьюлетт-Пак\-кард и устроил производство видеоверсии этих лекций.  Мы
хотели бы поблагодарить одаренных преподавателей --- в особенности Джима 
Миллера, Билла Сиберта и Майка Айзенберга, --- которые разработали
курсы повышения квалификации с использованием этих видеоматериалов и
преподавали по ним в различных университетах и корпорациях по всему миру.

Множество работников образования проделали значительную
работу по переводу первого издания. Мишель Бриан, Пьер Шамар и Андре
Пик сделали французское издание, Сюзанна Дэ\-ни\-елс-Хэ\-рольд выполнила
немецкий перевод, а Фумио Мотоёси~--- японский.  Мы не
знаем авторов китайского издания, однако считаем для себя честью быть
выбранными в качестве объекта <<неавторизованного>> перевода.

Трудно перечислить всех людей, внесших технический вклад в
разработку систем программирования на языке Scheme, которые мы
используем в учебных целях.  Кроме Гая Стила, в список важнейших
волшебников входят Крис Хансон, Джо Боубир, Джим Миллер,
Гильермо Росас и Стефен Адамс.  Кроме них, существенное время и силы
вложили Ричард Столлман, Алан Боуден, Кент Питман, Джон Тафт, Нил
Мэйл, Джон Лэмпинг, Гуин Оснос, Трейси Ларраби, Джордж Карретт, Сома
Чаудхури, Билл Киаркиаро, Стивен Кирш, Лей Клотц, Уэйн Носс, Тодд Кэсс, 
Патрик О'Доннелл, Кевин Теобальд, Дэниел Вайзе, Кеннет Синклер, Энтони 
Кортеманш, Генри М.~Ву, Эндрю Берлин и Рут~Шью.

Помимо авторов реализации MIT, мы хотели бы поблагодарить множество
людей, работавших над стандартом Scheme IEEE, в том числе Уильяма Клингера 
и Джонатана Риса, которые редактировали
R${}^{\mbox{4}}$RS, а также Криса Хэйнса, Дэвида Бартли,
Криса Хансона и Джима Миллера, которые подготовили стандарт
IEEE.

Долгое время Дэн Фридман был лидером сообщества языка
Scheme.  Работа сообщества в более широком плане переходит границы вопросов
разработки языка и включает значительные инновации в образовании,
такие как курс для старшей школы, основанный на EdScheme компании
Schemer's Inc. и замечательные книги Майка Айзенберга, Брайана Харви и 
Мэтью Райта. 

Мы ценим труд тех, кто принял участие в превращении
этой работы в настоящую книгу, особенно Терри Элинга, Ларри
Коэна и Пола Бетджа из издательства MIT Press. Элла Мэйзел нашла
замечательный рисунок для обложки.  Что касается второго издания, то
мы особенно благодарны Бернарду и Элле Мэйзел за помощь с оформлением
книги, а также Дэвиду Джонсу, великому волшебнику \TeX{}а.
Мы также в долгу перед читателями, сделавшими проницательные 
замечания по новому проекту: Джекобу Кацнельсону, Харди Мейеру, Джиму
Миллеру и в особенности Брайану Харви, который был для этой книги
тем же, кем Джули была для его книги {\em Просто
Scheme.}

Наконец, мы хотели бы выразить признательность организациям,
которые поддерживали нашу работу в течение этих лет.  Мы благодарны
компании Хьюлетт-Паккард за поддержку, которая стала возможной благодаря
Айре Гольдстейну и Джоэлю Бирнбауму, а также агентству
DARPA за поддержку, которая стала возможной благодаря Бобу Кану.\translationnote{Со
своей стороны хотелось бы поблагодарить Константина Добкина, Андрея Комеча, Сергея Коропа, Алексея Овчинникова, Алекса Отта, Вадима Радионова, Марию Рубинштейн и особенно Бориса Смилгу.}
